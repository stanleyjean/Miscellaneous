\documentclass[9pt, a4paper, twocolumn]{oblivoir}
\usepackage{fapapersize}
    \usefapapersize{210mm,297mm,20mm,20mm,25mm,25mm}
\usepackage{fancyhdr}
\pagenumbering{gobble}
\fancypagestyle{firststyle}
{
    \pagestyle{fancy}
% \fancyhf{}
% \rhead{Overleaf}
\lhead{\textbf{\LARGE Readings}}
% \rfoot{Page \thepage}

}

\pagestyle{firststyle}

\begin{document}
    
    \begin{enumerate}     
        \item 다음 글의 요지로 가장 적절한 것은?
        % 2021-고2 3월 22
        
        Fears of damaging ecosystems are based on the sound
        conservationist principle that we should aim to minimize the
        disruption we cause, but there is a risk that this principle
        may be confused with the old idea of a ‘balance of nature.’
        This supposes a perfect order of nature that will seek to
        maintain itself and that we should not change. It is a
        romantic, not to say idyllic, notion, but deeply misleading
        because it supposes a static condition. Ecosystems are
        dynamic,
        and
        although
        some
        may
        endure,
        apparently
        unchanged, for periods that are long in comparison with the
        human lifespan, they must and do change eventually. Species
        come and go, climates change, plant and animal communities
        adapt to altered circumstances, and when examined in fine
        detail such adaptation and consequent change can be seen to
        be taking place constantly. The ‘balance of nature’ is a
        myth. Our planet is dynamic, and so are the arrangements
        by which its inhabitants live together. 

        \begin{enumerate}
            \item Nature is always dynamically changing without being static. %%%%% 정답 
            \item Inappropriate intervention of humans will break the balance of ecosystems.
            \item Animals and plants keep the balance of ecosystems by competing with each other.
            \item Ecosystems are maintained through passive nature.
        \end{enumerate}
        % a
 
\pagebreak
        
        \item 다음 글의 주제로 가장 적절한 것은? 
        % 2021 고2 6월 23 

        Creativity is a step further on from imagination. Imagination
can be an entirely private process of internal consciousness.
You might be lying motionless on your bed in a fever of
imagination and no one would ever know. Private imaginings
may have no outcomes in the world at all. Creativity does.
Being creative involves doing something. It would be odd to
describe as creative someone who never did anything. To call
somebody creative suggests they are actively producing
something in a deliberate way. People are not creative in the
abstract; they are creative in something: in mathematics, in
engineering, in writing, in music, in business, in whatever.
Creativity involves putting your imagination to work. In a
sense, creativity is applied imagination.
\begin{flushright}
    {\small * deliberate: 의도적인 ** odd: 이상한}
\end{flushright}
    \begin{enumerate}
        \item the various meanings of imagination
        \item creativity as the realization of imagination
        \item diverse ways to enhance creativity of people 
        \item effects of a creative attitude on academic achievement
    \end{enumerate}

\pagebreak

        \item 다음 빈칸에 들어갈 말로 가장 적절한 것은? 
        % 2021 고2 6월31 

        The tendency for one purchase to lead to another one has
a name: the Diderot Effect. The Diderot Effect states that
obtaining a new possession often creates a spiral of
consumption that leads to additional purchases. You can spot
this pattern everywhere. You buy a dress and have to get
new shoes and earrings to match. You buy a toy for your
child and soon find yourself purchasing all of the accessories
that go with it. It’s a chain reaction of purchases. Many
human behaviors follow this cycle. You often decide what to
do next based on what you have just finished doing. Going
to the bathroom leads to washing and drying your hands,
which reminds you that you need to put the dirty towels in
the laundry, so you add laundry detergent to the shopping
list, and so on. No behavior happens in \rule{1cm}{0.1mm}. Each
action becomes a cue that triggers the next behavior.
\begin{flushright}
    {\small * cue: 신호, ** detergent: 세제}
\end{flushright}
    \begin{enumerate}
        \item mass
        \item harmony
        \item observation
        \item isolation %%%%%%%%%%%%%%%%% 정답 
    \end{enumerate}

\pagebreak
    \item 다음 글의 주제로 가장 적절한 것은?
    % 2021 고2 11월 23

    Shutter speed refers to the speed of a camera shutter. In
behavior profiling, it refers to the speed of the eyelid. When we
blink, we reveal more than just blink rate. Changes in the speed
of the eyelid can indicate important information; shutter speed
is a measurement of fear. Think of an animal that has a
reputation for being fearful. A Chihuahua might come to mind.
In mammals, because of evolution, our eyelids will speed up to
minimize the amount of time that we can’t see an approaching
predator. The greater the degree of fear an animal is
experiencing, the more the animal is concerned with an
approaching predator. In an attempt to keep the eyes open as
much as possible, the eyelids involuntarily speed up. Speed,
when it comes to behavior, almost always equals fear. In
humans, if we experience fear about something, our eyelids will
do the same thing as the Chihuahua; they will close and open
more quickly.
\begin{flushright}
    {\small * eyelid: 눈꺼풀, ** reputation: 평판\\ $***$ predator: 포식자 **** involuntarily: 비자발적으로}
\end{flushright}
    \begin{enumerate}
        \item eye contact as a way to frighten others
        \item fast blinking as a symptom of eye fatigue
        \item blink speed as a significant indicator of fear %%%% 정답 
        \item blinking eyes for predators leading to evolution  
    \end{enumerate}


\pagebreak
        \item 다음 빈칸에 들어갈 말로 가장 적절한 것은?
        % 2021 고2 9월 32 

        Philosophical activity is based on the \rule{1.4cm}{0.1mm}.
The philosopher’s thirst for knowledge is shown through
attempts to find better answers to questions even if those
answers are never found. At the same time, a philosopher
also knows that being too sure can hinder the discovery of
other and better possibilities. In a philosophical dialogue, the
participants are aware that there are things they do not
know or understand. The goal of the dialogue is to arrive at
a conception that one did not know or understand beforehand.
In traditional schools, where philosophy is not present,
students often work with factual questions, they learn
specific content listed in the curriculum, and they are not
required to solve philosophical problems. However, we know
that awareness of what one does not know can be a good
way to acquire knowledge. Knowledge and understanding are
developed through thinking and talking. Putting things into
words makes things clearer. Therefore, students must not be
afraid of saying something wrong or talking without first
being sure that they are right.
\begin{flushright}
    {\small * hinder: 저해하다, ** beforehand: 사전에}
\end{flushright}
    \begin{enumerate}
        \item recognition of ignorance %%%%%%% 정답 
        \item emphasis on self-assurance
        \item thinkers of preceding generation 
        \item comprehension of natural phenomena
    \end{enumerate}
\pagebreak
        \item 다음 글의 주제로 가장 적절한 것은? 
        % 2021-고2 3월 23

        Before the modern scientific era, creativity was attributed
to a superhuman force; all novel ideas originated with the
gods. After all, how could a person create something that
did not exist before the divine act of creation? In fact, the
Latin meaning of the verb “inspire” is “to breathe into,”
reflecting the belief that creative inspiration was similar to
the moment in creation when God first breathed life into
man. Plato argued that the poet was possessed by divine
inspiration, and Plotin wrote that art could only be beautiful
if it descended from God. The artist's job was not to imitate
nature but rather to reveal the sacred and transcendent
qualities of nature. Art could only be a pale imitation of the
perfection of the world of ideas. Greek artists did not
blindly imitate what they saw in reality; instead they tried
to represent the pure, true forms underlying reality,
resulting in a sort of compromise between abstraction and
accuracy.
\begin{flushright}
    \small{* transcendent: 초월적인 ** underlie: ~의 바탕이 되다.\\ $***$ inspire: 영감을 주다}
\end{flushright}
        \begin{enumerate}
            \item positive effect of imitation on creativity
            \item gods as a source of creativity in the pre-modern era %%%%%%%
            \item contribution of art to sharing religious beliefs
            \item inspiration as a source of religious belief 
        \end{enumerate}
    \end{enumerate}
\end{document}