\documentclass[a4paper, twocolumn]{oblivoir}
\usepackage{fapapersize}
    \usefapapersize{210mm,297mm,20mm,20mm,25mm,25mm}
\usepackage{fancyhdr}
\pagenumbering{gobble}
\fancypagestyle{firststyle}
{
    \pagestyle{fancy}
% \fancyhf{}
% \rhead{Overleaf}
\lhead{\textbf{\LARGE Readings}}
% \rfoot{Page \thepage}

}

\pagestyle{firststyle}
\begin{document}

    \begin{enumerate}
        \item 다음 글에서 필자가 주장하는 바로 가장 적절한 것은?
        
                Clarity in an organization keeps everyone working in one
        accord and energizes key leadership components like trust and
        transparency. No matter who or what is being assessed in
        your organization, what they are being assessed on must be
        clear and the people must be aware of it. If individuals in your
        organization are assessed without knowing what they are being
        assessed on, it can cause mistrust and move your organization
        away from clarity. For your organization to be productive,
        cohesive, and successful, trust is essential. Failure to have
        trust in your organization will have a negative effect on the
        results of any assessment. It will also significantly hinder the
        growth of your organization. To conduct accurate assessments,
        trust is a must — which comes through clarity. In turn,
        assessments help you see clearer, which then empowers your
        organization to reach optimal success.

        \begin{enumerate}
            \item 조직의 발전을 위해 구성원은 동료의 능력을 신뢰해야 한다
            \item 조직 내 구성원의 능력에 맞는 명확한 목표를 설정해야 한다.
            \item 조직의 신뢰 형성을 위해 구성원에 대한 평가 요소가 명확해야 한다.
            \item 구성원의 의견 수용을 위해 신뢰에 기반한 조직 문화가 구축되어야 한다.
            % 3번 정답 
            % 고2 2022년도 11월 시행 
        \end{enumerate}
        
        \pagebreak

        \item 다음 글의 요지로 적절한 것은? 
        
                In one study, when researchers suggested that a date was
        associated with a new beginning (such as “the first day of
        spring”), students viewed it as a more attractive time to
        kick-start goal pursuit than when researchers presented it as
        an unremarkable day (such as “the third Thursday in March”).
        Whether it was starting a new gym habit or spending less time
        on social media, when the date that researchers suggested was
        associated with a new beginning, more students wanted to begin
        changes right then. And more recent research by a different
        team found that similar benefits were achieved by showing goal
        seekers modified weekly calendars. When calendars depicted
        the current day (either Monday or Sunday) as the first day of
        the week, people reported feeling more motivated to make
        immediate progress on their goals.
        \begin{flushright}
            \small{* procrastinate: 미루다}
        \end{flushright}

        \begin{enumerate}
            \item 자신이 해야 할 일을 일정표에 표시하는 것이 목표 달성에 효과적이다.
            \item 문제 행동을 개선하기 위해 원인이 되는 요소를 파악할 필요가 있다. 
            \item 타인이 시작 날짜를 제시해줄 때 더 확실하게 목표를 달성할 수 있다.
            \item 날짜가 시작이라는 의미와 관련지어질 때 목표 추구에 강한 동기가 부여된다.
            % 4번 정답 
            % 고2 2022년도 11월 시행 

            \pagebreak
        
        \end{enumerate}
        \item 다음 글의 제목으로 가장 적절한 것은?
   
        A building is an inanimate object, but it is not an
        inarticulate one. Even the simplest house always makes a
        statement, one expressed in brick and stone, in wood and
        glass, rather than in words — but no less loud and obvious.
        When we see a rusting trailer surrounded by weeds and
        abandoned cars, or a brand-­new mini-­mansion with a high
        wall, we instantly get a message. In both of these cases,
        though in different accents, it is “Stay Out of Here.” It is
        not only houses, of course, that communicate with us. All
        kinds of buildings — churches, museums, schools, hospitals,
        restaurants, and offices — speak to us silently. Sometimes
        the statement is deliberate. A store or restaurant can be
        designed so that it welcomes mostly low-­income or
        high-­income customers. Buildings tell us what to think and
        how to act, though we may not register their messages
        consciously.
        \begin{flushright}
            \small{* inarticulate: 표현을 제대로 하지 못하는}
        \end{flushright}
        \begin{enumerate}
            \item Buildings Do Talk in Their Own Ways!
            \item Design of Buildings Starts from Nature
            \item Language of Buildings: Too Vague to Grasp
            \item How Buildings Have Emotions? 
            % 답 1번 
            % 고2 2022년 6월 시행 
        \end{enumerate}
        \pagebreak

        \item 다음 글의 요지로 가장 적절한 것은?
        
                The problem with simply adopting any popular method of
        parenting is that it ignores the most important variable in
        the equation: the uniqueness of your child. So, rather than
        insist that one style of parenting will work with every child,
        we might take a page from the gardener’s handbook. Just as
        the gardener accepts, without question or resistance, the
        plant’s requirements and provides the right conditions each
        plant needs to grow and flourish, so, too, do we parents
        need to custom­design our parenting to fit the natural needs
        of each individual child. Although that may seem difficult, it
        is possible. Once we understand who our children really are,
        we can begin to figure out how to make changes in our
        parenting style to be more positive and accepting of each
        child we’ve been blessed to parent.
        \begin{flushright}
            \small{* equation: 방정식}     
        \end{flushright}
        \begin{enumerate}
            \item 정서적 교감은 자녀의 바람직한 인격 형성에 필수적이다. 
            \item 정원을 잘 가꾸기 위해서는 적절한 조건이 필요하다. 
            \item 자녀의 특성에 맞는 개별화된 양육이 필요하다. 
            \item 아이들을 잘 양육하기 것에 정원 가꾸기가 도움이 된다. 
            % 답 3번 
            % 고2 2022년 6월 시행 
        \end{enumerate}

        \pagebreak
        \item 다음 빈칸에 들어갈 말로 가장 적절한 것은?
        
                Followers can be defined by their position as subordinates
        or by their behavior of going along with leaders’ wishes. But
        followers also have power to lead. Followers empower
        leaders as well as vice versa. This has led some leadership
        analysts like Ronald Heifetz to avoid using the word
        followers and refer to the others in a power relationship as
        “citizens” or “constituents.” Heifetz is correct that too simple
        a view of followers can produce misunderstanding. In modern
        life, most people wind up being both leaders and followers,
        and the categories can become quite \rule{1cm}{0.1mm}. Our behavior
        as followers changes as our objectives change. If I trust
        your judgment in music more than my own, I may follow
        your lead on which concert we attend (even though you may
        be formally my subordinate in position). But if I am an
        expert on fishing, you may follow my lead on where we
        fish, regardless of our formal positions or the fact that I
        followed your lead on concerts yesterday.
        \begin{flushright}
            \small{* vice versa: 반대 또한 같은}     
        \end{flushright}
        \begin{enumerate}
            \item unfair
            \item stable
            \item fluid
            \item solid
            %답 3번 
            % 고2 2022년 6월 시행
        \end{enumerate}
    \end{enumerate}
\end{document}