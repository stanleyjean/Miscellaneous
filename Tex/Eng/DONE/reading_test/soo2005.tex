\documentclass[a4paper, twocolumn]{oblivoir}
\usepackage{fancyhdr}
\usepackage{fapapersize}
    \usefapapersize{210mm,297mm,20mm,20mm,25mm,18mm}
\usepackage{enumitem}

\pagestyle{fancy}
\fancyhf{}
% \rhead{Overleaf}
\lhead{\textbf{\Huge Reading Exercises}}

\begin{document}
    \noindent
    \textbf{1. 다음 빈칸에 가장 적절한 것을 고르시오.}\\

    The number of foreigners interested in the Korean
    language has increased dramatically over the past few
    years because of the success of Korean firms overseas
    and growing interest in Korean culture. For example,
    many Chinese students have become interested in Korean
    as they plan to work for Korean firms, which offer
    better opportunities and pay. The total number of foreign
    students attending Korean language programs has
    increased to more than 30,000 in Seoul alone this year
    from about 4,700 at the end of last year. People speaking
    Korean have long been limited mostly to those from the
    peninsula. It is no wonder few people ever imagined that
    the country s language might one day \rule{1.2cm}{0.15mm}.
    \begin{flushright}
        \small{* overseas: 해외로 ** peninsula: 반도}
    \end{flushright}

   
\noindent 
① provide some enthusiastic technical support \\
② open new opportunities for its modern art\\
③ remain one of the most scientific languages\\
④ contribute to the return of its ancient culture\\
⑤ become popular in the international community\\

    \noindent
    \textbf{2. 다음 글의 요지로 가장 적절한 것을 고르시오.}\\

    How much one can earn is important, of course, but
    there are other equally important considerations, neglect of
    which may produce frustration in later years. Where there
    is genuine interest, one may work diligently without even
    realizing it, and in such situations success follows. More
    important than success, which generally means promotion
    or an increase in salary, is the happiness which can only
    be found in doing work that one enjoys for its own sake
    and not merely for the rewards it brings.
    \begin{flushright}
        \small{* for one's own sake: $\sim $ 스스로를 위해서}
    \end{flushright}

    
    \noindent
    ① 성공하기 위해서는 성실한 자세가 필요하다.\\
② 일의 즐거움에서 얻는 행복이 중요하다.\\
③ 개인의 이익보다 전체의 이익이 우선한다.\\
④ 성공하면 그에 상응하는 보상이 뒤따른다.\\
⑤ 승진을 위해서는 철저한 자기 관리가 필요하다.\\

\pagebreak
\noindent
\textbf{3. 다음 글의 주장으로 가장 적절한 것을 고르시오.}\\

Nowadays, we can enjoy athletic competition of every
kind without leaving our homes. It is the fun that comes
from cheering on our team and celebrating its skills
while complaining about the opposing team’s good luck.
But some individuals sit and watch a football game or
tennis match without cheering for anyone or any team.
They are not willing to risk the possible disappointment
of picking the loser, so they give up the possible joy of
picking the winner. They live in the world of neutrality.
Don’t be one of them. Sure, your team might lose. But
then again, your team might win. Either way, your
match-watching experience will have been a fun one, and you
will have avoided being merely a passive observer.
\begin{flushright}
    \small{* neutrality: 중립성 ** stimulate: 자극하다}
\end{flushright}

\noindent
① praise the win of the opposing team.\\
② enjoy the game at stadium.\\
③ pick a team to cheer up on to enjoy a match.\\
④ be neutral to analyse a match.\\
⑤ don't stimulate opponents' supporters with cheering\\


\noindent
\textbf{4. 다음 빈칸에 가장 적절한 것을 고르시오.}\\

There are some people who believe that no one should
be trusted. They usually feel this way because their
behavior forces others to lie to them. In other words,
they make it difficult for others to tell them the truth
because they respond rudely or emotionally to people who
tell the truth. If others see how angry, hurt, or hateful you
become when they tell you the truth, they will avoid telling
it to you at all costs. If you are known as someone who is
easily offended, you will never know what others are really
thinking or feeling because they will \rule{1.2cm}{0.15mm}
to escape from your negative reaction. If you demand that
children tell you the truth and then punish them because it
is not very satisfying, you teach them to lie to you to
protect themselves.\\

\noindent
① protect their children\hspace{1em} ② change the truth\\
③ waste your expenses \hspace{1em} ④ hurt your feelings\\
⑤ show their anger\\


\pagebreak

\begin{enumerate}
    \item ⑤
    \item ②
    \item ③
    \item ②
\end{enumerate}




\end{document}