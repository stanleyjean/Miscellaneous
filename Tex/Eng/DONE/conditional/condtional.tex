\documentclass[ a4paper]{oblivoir}
\usepackage{fancyhdr}
\usepackage{fapapersize}
    \usefapapersize{210mm,297mm,20mm,20mm,25mm,25mm}
\usepackage{tabularray}
\usepackage{enumitem}
\usepackage{cancel}
\usepackage{caption}
\usepackage{amsmath}
\usepackage{fancyhdr}
% \fancyhf{}
% \renewcommand{\headrulewidth}{0pt}
% \fancyfoot[c]{}
% \fancypagestyle{plain}{
% \lfoot{\copyright\:2022. 손민우 All rights reserved} 
% }
% \pagestyle{plain}

% \fancypagestyle{FirstPage}{
% \lfoot{\copyright 2022. 손민우 All rights reserved} 
% }


% \usepackage{multicol}
% \setlength{\columnseprule}{0.01pt}
\pagenumbering{gobble}
\captionsetup{labelformat = empty}
\begin{document}
    \chapter*{\Huge 조건문}
    
    조건문은 직설법과 가정법으로 나뉘어진다. 
    총 네 가지 종류의 조건문이 존재함을 알아야한다. (주절과 조건절의 동사 시제를 유의하여 볼 것)
    \begin{enumerate}
        \item If the company \emph{fails}, we \emph{lose} our money. 
        \item If the company \emph{fails}, we \emph{will lose} our money. 
        \item If the company \emph{failed}, we \emph{would lose} our money. 
        \item If the company \emph{had failed}, we \emph{would have lost} our money. 
    \end{enumerate}

    
    열려 있는 상황(open condition, 발생 가능성이 존재)에는 직설법을 사용하고, 실현불가능하거나 일어날 가능성이 적은 상황에서는
    가정법을 사용한다. 아래 두 조건절을 비교해보라: 
    \begin{itemize}
        \item if you \emph{go} to the agencies
        \item if you \emph{lived} on the planet Mercury
    \end{itemize}

    
    {\footnotesize{$\dagger$ 참고 

    if 조건절은 주절의 앞이나 뒤에 올 수 있다:
    \begin{itemize}
        \item \emph{If you're in a hurry}, you don't need to wait for me.
        \item You don't need to wait for me \emph{if you're in a hurry}.
    \end{itemize}
    }}

  %  \pagebreak

    \section{직설법(Type 0, 1)}
    \begin{itemize}
        \item Type 0: 현재 + 현재, 특정한 일이 발생하면 뒤의 일은 ``자동적으로''  따라오는 경우에 사용한다.
            \begin{itemize}
                \item If the doorbell \emph{rings}, the dog \emph{barks.} 
                \item If I \emph{go} to school earlier than usual, I always \emph{meet} the girl.
                \item \emph{If/When} I wake up, I always hear my dog barking.
            \end{itemize}
        \item Type 1: 현재 + will(미래), 매우 자주 사용되는 형태이고, 일반적인 조건 상황을 나타낸다. 
        (will 대신 shall도 가능)
            \begin{itemize}
                \item If it \emph{rains}, we \emph{won't clean} our frontyard.
                \item If \emph{have} enough time, I \emph{will watch} later tonight. 
                \item If you \emph{are leaving}, the result will be horrible.
                \item If you \emph{haven't gotten} television, you \emph{can't watch} it.
                \item If you don't decide right now, you \emph{may take} some disadvantages.
                \item If you make a mistake, \emph{don't panic}.
            \end{itemize}
            현재 시제의 표현으로 미래 시제를 나타낼 수 있다는 점을 이해해야한다. (e.g. Let's wait until everyone arrives.)
            
            {\footnotesize $\dagger$ 참고
            
            \begin{enumerate}
            
            \item 일반적으로 if 조건문 내에 will 동사를 사용해서는 안된다. 그러나, If 조건문 내에 의지를 표현하거나 요구를 나타내기 위해 will을 사용하거나 거절을 뜻하기 위해 won't을 사용할 수도 있다. 
            \begin{itemize}
                \item If all of you \emph{will lend} me a hand, we'll soon get the job done. $\triangleright$ 적극성
                \item If you \emph{will take} a seat, someone will be with you in a moment. $\triangleright$ 요구
                \item If the car \emph{won't start}, I'll have to ring the garage. $\triangleright$ 거절, 어려움
            \end{itemize}
            \item  비격식체로 if 조건절을 사용하지 않고도 조건문 표현이 가능하다.
            \begin{itemize}
                \item Touch me \emph{and} I'll scream.
                \item Leave the house \emph{or} you will get hurt.
            \end{itemize}
        \end{enumerate}}
    \end{itemize}
    \section{가정법 과거(Type 2)}

    과거 + would 의 형태, 비현실적인 상황의 가정이나 상상을 표현할 때 사용한다. 그 외에도 이론적인 가능성을 표현할 수 있다. 
    \begin{itemize}[label = {--}]
        \item If I \emph{had} lots of money, I \emph{would travel} round(around) the world.
        \item I\emph{'d tell} you the answer if I \emph{knew} what it was. 
        \item If \emph{were living} more diligiently, I \emph{would feel} much better and productive.
        \item If we \emph{caught} the early train tomorrow, we\emph{'d be} in York by lunch time.
        \item If the sun \emph{was shining}, everything would be perfect.
        \item if \emph{could have finished} the work earlier, I would watch the game.
        \item If I were with my friend here, I \emph{could(or might)} talk to him about this problem.
    
    \end{itemize}

    \begin{itemize}[label = {}]
        \item {\footnotesize $\dagger$ 참고

        \begin{enumerate}
            \item 직설법을 사용하지 않고 가정법을 사용했을 때 공손하게 표현이 가능한 상황이 존재한다. 
            \item 아래 경우의 가정법 과거도 사용이 가능하다. would 를 if 조건절에 사용함으로써 요구를 나타낼 수도 있다.
                \begin{itemize}[label = {--}]
                    \item If I had lots of money, I \emph{would/should} travel round the world. $\triangleright$ would를 사용이 훨씬 더 일반적
                    \item If I \emph{was/were} here, I wouldn't be able to see this scenery now.
                    \item If you\emph{'d sign} on this, please.
                    \item If you \emph{would like} to see the exhibition, it would be nice to go together.
                \end{itemize}
                
        \end{enumerate}}
    \end{itemize}
    

    
    \section{가정법 과거완료(Type 3)}

    과거완료 + would + 완료, 실현 불가능한 ``과거'' 행동에 대해서 이야기할 때 사용한다. 
    
    \begin{itemize}[label = {--}]
        \item If you \emph{had taken} a taxi, you \emph{would have got} here on time.
        \item If I \emph{had passed} the test, I \emph{wouldn't have missed} the chance to buy the product.
        \item If I \emph{could have remembered} the meeting, I wouldn't have missed it.
        \item If she had chosen the right path, she \emph{could(or might)} have experienced less pain and agony.
        \item If he hadn't been kicked out of the house, he \emph{wouldn't have been sleeping} at my house.
        
    \end{itemize}
    
    \begin{itemize}[label = {}]
        \item {\footnotesize $\dagger$ 참고
        
        if 조건절에 would를 사용하거나 주절에 would를 포함하지 않은 과거 완료 표현을 사용하지 않도록 한다.
        
        다른 조건문의 형태와 섞어 사용할 수 있음에 주의한다. 
        \begin{itemize}[label = {--}]
            \item If John \emph{was} ambitious, he \emph{would} have found a way to make it work for his career. $\triangleright$ if 과거 + would 완료
            \item If you are good at math, you \emph{should have got} the perfect score in the test. $\triangleright$ if 현재 + would 완료
        \end{itemize}
        }
    \end{itemize}
    

    \section{Etc.}

    \subsection{wish, if only}

    \begin{enumerate}
        \item wish + would, 미래의 변화에 대한 소망/갑작스러운 요구나 불평
        \begin{itemize}[label = {--}]
            \item I wish my daughter \emph{wouldn't give up} in the next chance.
            \item I wish you \emph{would write} a letter to your parents.
        \end{itemize}
        \item wish + 과거/could, 현재의 상황에 대한 소망 ! would 랑은 쓸 수 없음 !
        \begin{itemize}[label = {--}]
            \item I wish she \emph{loved} me still.
            \item I wish he \emph{could be} here with me.
        \end{itemize}
        \item wish + 과거완료/could have p.p., 과거에 대한 소망 ! would랑은 쓸 수 없음 ! 
        \begin{itemize}[label = {--}]
            \item I wish Peter \emph{had} never \emph{seen} this horrible situation. 
            \item I wish you \emph{'d given} me a second chance to give it a go.
        \end{itemize}
        \item If only, wish를 이용한 조건문과 유사하게 사용한다. 
        \begin{itemize}[label = {--}]
            \item If only you \emph{would write} a letter to your parents. 
            \item If only she \emph{loved} me still.
            \item If only you\emph{'d given} me a second chance. - 강조를 강화하기 위해 wish 대신 사용가능
            \item If I'd \emph{only} thought it through, I wouldn't have made such a mistake. - only가 중간에 들어갈 수도 있음 
        \end{itemize}
    \end{enumerate}
    \subsection{should, were}

    if 조건절이 표현하는 상황이 실현 가능함을 나타내기 위해서 should나 happen to를 사용할 수 있다.
    \begin{itemize}[label = {--}]
        \item If you \emph{should} fall ill, the company will pay your hospital expenses. $\triangleright$ 직설법(Type 1)
        \item If I \emph{should} be chosen, I would do my best. $\triangleright$ 가정법 과거(Type 2)
        \item If you (should) \emph{happen to} pass the test, I will give you a reward for you result.  
    \end{itemize}  

    \begin{itemize}[label = {--}]
        \item If the picture was/\emph{were} genuine, it would be worth loads of money. 
        \item If the decision \emph{were to} go against us, we would appeal.
    \end{itemize}


    \subsection{도치}


    \begin{itemize}[label = {--}]
        \item \emph{Should} you fall ill, the company will pay your hospital expenses.
        \item \emph{Were} the picture genuine, it would be worth loads of money.
        \item \emph{Had} you taken a taxi, you would have gotten here on time.
    \end{itemize}

    {\footnotesize $\ast$ 주의 was는 가정법 조건절을 도치할 때 사용할 수 없다.}

    \subsection{기타 조건문}
    \begin{itemize}[label = {--}]
        \item If the figures aren't correct, then somebody must have made a mistake.
        \item \emph{If it hadn't been for you}, I'd have suffocated.
        \item \emph{But for} you, I'd have suffocated. (위의 문장과 비교해보라)
        \item I wouldn't be doing this job \emph{if it wasn't/weren't for my parents' requests}. \\
            then을 조건문에 사용하게 되면, 조건절과 주절의 ``연결성''을 강조하게 된다. (조건의 충족 여부와 결과 발생 사이의 관계에 유의)
    \end{itemize}


    
    이 외에도
    \begin{enumerate}
     \item (if + 형용사)로 나타내어 분사구문과 유사한 꼴을 갖는 조건문
     \item what if를 이용한 조건문
     \item even if/though를 이용한 조건문 (even if와 even though를 사용하는 경우의 차이는 무엇일까?)
     \item unless를 활용한 조건문
     \item as long as, provided, in case, with/witout/but for 등을 이용한 조건문
    \end{enumerate}
    이 있다. 



\end{document}