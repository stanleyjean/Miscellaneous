\documentclass[11pt, twocolumn, a4paper]{oblivoir}
\usepackage{fancyhdr}
\usepackage{fapapersize}
    \usefapapersize{210mm,297mm,20mm,20mm,25mm,25mm}
\usepackage{tabularray}
\usepackage{enumitem}
\usepackage{cancel}
\usepackage{caption}
\usepackage{amsmath}
\usepackage{fancyhdr}

\begin{document}
    \chapter*{영어 문장 구성의 원리}

    
    알파벳(a, b, c, ...) $\rightarrow$ 단어 or 구/절 $\rightarrow$ 문장
    
    \chapter*{내용 정리}
   
  \noindent  \emph{아래 수업 내용에 대해서, 각 내용에 대한 예시를 제시할 수 있어야 한다. }

    \section*{영어 문장 구성}

    \textbf{문장 성분의 뜻}
    \begin{itemize}
        \item S:
        \item V:
        \item O:
        \item C:
        \item A:
    \end{itemize}

    \textbf{문장 구성}
    \begin{itemize}
        \item S + V
        \item S + V + C
        \item S + V + A
        \item S + V + O
        \item S + V + I.O. + D.O.
        \item S + V + O + O.C.
        \item S + V + O + A
    \end{itemize}

    \pagebreak
    \section*{품사}
    \begin{itemize}
        \item 동사
        \begin{itemize}
            \item $[\qquad]$동사와 $[\qquad]$동사로 나뉜다. 
            \item 시제: 
            \item 동사의 종류별 부정문, 의문문
        \end{itemize}
        \item 명사
        \begin{itemize}
            \item 대명사
            \item 셀 수 있는 vs 셀 수 없는: 복수형 명사는 어떤 명사? 
            \item a/an, the는 언제 쓰일까? 
        \end{itemize}
        \item 형용사
        \begin{itemize}
            \item 명사 꾸미기
            \item 보어
        \end{itemize}
        \item 부사
        \begin{itemize}
            \item 부사, 동사, 형용사, 문장 전체 수식
            \item 위치/시간, 빈도, 정도를 나타냄 
        \end{itemize}
        \item 전치사
        \begin{itemize}
            \item 장소, 시간 등을 나타낼 때 쓰임 
            \item $[\ \ \ \ \ ]$가 뒤에 온다. 
        \end{itemize}
    \end{itemize}

    \section*{의문사}
    \begin{itemize}
        \item who
        \item which, what
        \item when, where, why
        \item how
    \end{itemize}

    \section*{문장의 유형}
    \begin{itemize}
        \item 의문문
        \item 명령문
        \item 청유문
        \item 부가의문문
    \end{itemize}
    
    \section*{기타}
\noindent    \textbf{비교급/최상급}

\noindent     $[\qquad]$사 

    \emph{good - better - best}

    \emph{bad - worse - worst}

    \emph{far - farther/further - furthest}
\newline
    $[\qquad]$사

    \emph{well - better - best}

    \emph{badly - worse - worst}

    \emph{far - farther/further - furthest}
    \newline
    수량 비교급 기본 형태 

    \emph{more - most}

    \emph{fewer - fewest}

    \emph{less - least}
    \begin{itemize}
        \item 비교급의 표현 두가지 
        \begin{itemize}
            \item -r/-er 
            \item more + 원급(형용사/부사)
        \end{itemize}
        \item 최상급의 표현 두 가지 
        \begin{itemize}
            \item the + -est
            \item the most + 원급(형용사/부사)
        \end{itemize}
    \end{itemize}

    \textbf{비교하는 방식}
  
    \begin{itemize}[label = {}]
        \item 
        
    \begin{itemize}
        \item 비교급 비교: 비교급 + $[\qquad]$을 이용한다
        \item 원급 비교: $[\qquad]$ + 원급 + $[\qquad]$을 이용한다
    \end{itemize}
\end{itemize}
\pagebreak
    \textbf{접속사}

    서로 다른 내용의 두 개 문장을 연결할 때 사용한다. 
    \begin{itemize}
        \item and, or, but
        \item before, after 
        \item when
        \item because
    \end{itemize}
    

\end{document}

