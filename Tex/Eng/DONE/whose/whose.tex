\documentclass[a4paper]{oblivoir}
\usepackage{fancyhdr}
\usepackage{fapapersize}
    \usefapapersize{210mm,297mm,20mm,20mm,25mm,25mm}
\usepackage{enumitem}
\usepackage{cancel}


\begin{document}


\noindent
    \textbf{\LARGE 한정사}\\

    \noindent 어떤 명사가 특정될 수 있도록 하며 명사를 수식하는 품사. 명사 하나에는 한정사 하나만 수식가능하다.
    \\ 두 단어 이상으로 이루어지는 경우도 존재한다(e.g., a number of, a few, loads of, a lot of)
    \begin{itemize}[label = {--}]
        \item 관사: a/an, the, \ldots
        \item 수량한정사: many, few, little, \ldots
        \item 지시한정사: this, that, such, \ldots
        \item 소유한정사: my, his, her, \ldots
    \end{itemize} 
\rule{\textwidth}{0.4pt}
\vskip 1cm
\noindent
\textbf{\LARGE Whose의 용법}

    \begin{enumerate}
        \item \emph{whose}는 소유격의 의미를 가짐
              \\We stopped to help some people \textbf{whose} car had broken down.
              \\(\textbf{Their} car had broken down.)

              관계사절에서는 \emph{whose}를 명사앞에 한정사로서 사용한다(e.g., whose car).
              \\ \xcancel{some people whose the car had broken down}
        \item \emph{whose} + (명사)는 관계사절의 목적어나 주어가 될 수 있다. 
              \\ Doctors are people \textbf{\emph{whose work}}  is obviously useful.
              \\ The prize goes to the contestant \textbf{\emph{whose performance}} TV viewers like best.

              \emph{whose} + (명사)는 전치사의 목적어가 될 수도 있다. 
              \\ I wish to thank all those people \textbf{\emph{without whose}} help I would 
              never had got(ten) this far.  
              \\ My best friend was Martin, \textbf{\emph{at whose}} wedding I had first met my future wife.
              \\ The neighbour \textbf{\emph{whose}} dog I'm looking \textbf{\emph{after}} is in Australia.

              \emph{whose} 는 계속적 용법에서도 사용이 가능하다. 
              \\ The ball fell to Collins, \textbf{\emph{whose shot}} hit the post. 

        \item \emph{whose}는 일반적으로 사람과 관계된다: some \textbf{\emph{people whose}} car had broken down.
              \\ 직접적으로 사람과 관계되지 않은 명사와도 사용될 수 있다. 특히, 사람으로 구성된 조직이나 사람의 활동을 일컫을 때 사용가능하다. 
              \\ It's the poorer \textbf{\emph{countries whose}} exports are earning less money.
              \\ I wouldn't fly with an \textbf{\emph{airline whose}} safety record is so bad.
              \\ She sang a beautiful \textbf{\emph{song}}, \textbf{\emph{whose}} sentiments moved to the audience
        \item \emph{whose}로 선행사와 연결하지 않고 the + (명사) + of which의 형태로 쓸 수 있다. 
              \\ She sang a beautiful song, \textbf{\emph{the sentiments of which}} moved the audience.
              \\ We are introducing a new system, \textbf{\emph{the aim of which}} is to cut costs.
              \\ You should look up any word \textbf{\emph{the meaning of which}} is unclear.

              $\diamond $ 참고
              \\ \lbrack The + 명사\rbrack는 \emph{of which} 다음에 올 수 있다. 
              \\  You shoule look up any word \textbf{\emph{of which the meaning}} is unclear.

    


    \end{enumerate}
\end{document}