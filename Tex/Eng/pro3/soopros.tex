\documentclass[9pt, a4paper, twocolumn]{oblivoir}
\usepackage{fapapersize}
    \usefapapersize{210mm,297mm,20mm,20mm,25mm,25mm}
\usepackage{fancyhdr}
\pagenumbering{gobble}
\fancypagestyle{firststyle}
{
    \pagestyle{fancy}
% \fancyhf{}
% \rhead{Overleaf}
\lhead{\textbf{\LARGE Readings}}
% \rfoot{Page \thepage}

}

\pagestyle{firststyle}

\begin{document}
    \begin{enumerate}
        \item 다음 글의 제목으로 가장 적절한 것은? 
        % 2012 06 42번 

        Whenever you stand on a scale in your bathroom or place a
melon on a scale at the grocery store, you are measuring
weight. An object’s weight is the force exerted on it by
gravity, usually the earth’s gravity. When you stand on a
bathroom scale, the scale measures just how much upward
force it must exert on you in order to keep you from moving
downward toward the earth’s center. As in most scales you will
encounter, the bathroom scale uses a spring to provide this
upward support. If you are stationary, you are not accelerating,
so your downward weight and the upward force from the
spring must cancel one another; that is, they must be equal in
magnitude but opposite in direction so that they sum to zero
net force.
\begin{flushright}
    {\small $*$ exert 행사하다}
\end{flushright}
        \begin{enumerate}
            \item Selecting a Good Scale
            \item The Best Way to Measure Your Weight
            \item Weight: Two Forces in Balance %%%%
            \item The Earth's Gravity: A Mysterious Power
            \item How to Control your Weight
        \end{enumerate}

        \pagebreak

        \item 다음 빈칸에 들어갈 말로 가장 적절한 것은?
        
        %2014 예비시행 33번 
        During the hundreds of millions of years that plants
have been living on our planet, they have become
amazingly self-sufficient. In addition to establishing a
useful relationship with the sun, plants have learned \rule{5cm}{0.1mm}
. When plants die, they seem
to just fall on the ground and rot, getting eaten by many
bugs and worms. However, researchers were shocked to
discover that dead plants get consumed only by particular
bacteria and fungi. Plants know how to attract to their
own rotting only those microorganisms and earthworms
that will produce beneficial minerals for the soil where
the plants’ siblings will grow. One way plants attract
particular~microorganisms~into~their~soil~is~by
concentrating more sugars in their roots. Thus roots such
as carrots and potatoes are always much sweeter than the
rest of the plant. Apparently, the quality of the soil is
critically important, not only as a source of water and
minerals for plants but for their very survival.

    \begin{enumerate}
        \item to extend their lifespan
        \item to grow their own soil %%%
        \item to consume microorganisms
        \item to survive sttacks of bacteria
        \item to keep the environment clean
    \end{enumerate}
    \pagebreak
        \item 다음 빈칸에 들어갈 말로 가장 적절한 것은?
        
        % 2012 6월 27번 
        Some people believe that \rule{2cm}{0.1mm}
is some kind
of instinct, developed because it benefits our species in some
way. At first, this seems like a strange idea: Darwin’s theories
of evolution presume that individuals should act to preserve
their own interests, not those of the species as a whole. But
the British evolutionary biologist Richard Dawkins believes
that natural selection has given us the ability to feel pity for
someone who is suffering. When humans lived in small
clan-based groups, a person in need would be a relative or
someone who could pay you back a good turn later, so taking
pity on others could benefit you in the long run. Modern
societies are much less close-knit and when we see a heartfelt
appeal for charity, chances are we may never even meet the
person who is suffering ― but the emotion of pity is still in
our genes.
\begin{flushright}
    {\small $*$ preserve: 보존하다 $**$ clan: 씨족, 집단 \\ $***$ charity: 자선}
\end{flushright}
\begin{enumerate}
    \item not wanting to suffer 
    \item giving to charity %%%
    \item drawing pity from others 
    \item exploring alternatives
    \item pursuing individual interests 
\end{enumerate}

\pagebreak

    \item 다음 빈칸에 들어갈 말로 적절한 것은?

% 2011 수능 24번 
In a classic set of studies over a ten-year period, biologist
Gerald Wilkinson found that, when vampire bats return to
their communal nests from a successful night’s foraging, they
frequently vomit blood and share it with other nest-mates,
including even non-relatives. The reason, it turns out, is that
blood-sharing greatly improves each bat’s chances of survival.
A bat that fails to feed for two nights is likely to die.
Wilkinson showed that the blood donors are typically sharing
their surpluses and, in so doing, are saving unsuccessful
foragers that are close to starvation. So the costs are
relatively low and the benefits are relatively high. Since no
bat can be certain of success on any given night, it is likely
that the donor will itself eventually need help from some
nest-mate. In effect, the vampire bats have created a kind
of \rule{3.5cm}{0.1mm}. 
\begin{flushright}
    {\small $*$ forage: 먹이를 찾아다니다 \\ $**$ surplus: 잉여, 흑자, $***$ stravation: 굶주림}
\end{flushright}

    \begin{enumerate}
        \item complex social hierarchy
        \item ecological diversity
        \item mutual insurance system %%
        \item parasitic relationship
        \item effective reproduction process
    \end{enumerate}

    \end{enumerate}
    


\end{document}