\documentclass[a4paper, 12pt, twocolumn]{oblivoir}
\usepackage{fapapersize}
    \usefapapersize{210mm,297mm,20mm,20mm,25mm,25mm}
\pagenumbering{gobble}
\begin{document}
    \section{기본 문장 형식} % 명령문, 청유문 등도 넣기
    \begin{enumerate}
        \item He runs fast.
        \item There are chairs.
        \item The house has many windows. 
        \item I live in Korea.
        \item Let's go to his house.
        \item He ate a piece of pizza. 
        \item Peter is a student.
        \item Don't use your phone. 
        \item Music gives us good feelings.
        \item I opened the door. 
        \item Be polite to others. 
        \item The girl is pretty. 
        \item Jack looks nice today. 
        \item My mother went out this morning. 
        \item The movie made me feel happy. 
    \end{enumerate}


    \section{동사} % 일반 동사, be 동사, 시제 내용과 조동사도 합쳐서 
    \subsection{의문문}
    \begin{enumerate}
        \item Will she come home today? 
        \item Is he healthy now? 
        \item Was it rainy yesterday?
        \item Did he make a mistake? 
        \item Are you busy today? 
        \item Does he live in Seoul? 
    \end{enumerate}
    \subsection{부정문}
    \begin{enumerate}
        \item Peter is not a student. 
        \item The movie wasn't(was not) scary.
        \item The cookies are not sweet. 
        \item My brother doesn't drink milk 
        \item It won't be(will not) fun. 
        \item Jack didn't make a mistake. 
    \end{enumerate}

    \subsection{의문문 + 부정문}
    \begin{enumerate}
        \item Didn't you study yesterday?
        \item Isn't he Peter? 
        \item Won't you stop making noise?
    \end{enumerate}
    \section{다른 품사들} % 명사, 형용사, 부사, 전치사 
    \subsection{형용사}
    \begin{enumerate}
        \item The library has a lot of great books. 
        \item The man is alone in the room. 
        \item It looked nice and pretty. 
    \end{enumerate}
    \subsection{부사}
    \begin{enumerate}
        \item He is always having fun.
        \item Do you study often?
        \item Tom is very lazy. 
        \item My team won the game luckily.
        \item I have to leave early. 
    \end{enumerate}
    \subsection{전치사}
    \begin{enumerate}
        \item I have lunch at noon.
        \item Grace live here for three years. 
        \item I am in the town now. 
        \item Look at the ceiling.
    \end{enumerate}
    \section{의문사}
    \begin{enumerate}
        \item What is the matter?
        \item How was your weekend?
        \item Where will you go?
        \item When are you going to leave?
        \item Who did you meet last night?
        \item What kind of movies do you like?
        \item How many people live here? 
        \item Where is your school? 
        \item Why were you late? 
        \item Who did the job? 
    \end{enumerate}
    \section{준동사 \normalsize{(to 부정사, 동명사)}} % to 부정사, 동명사 
    \begin{enumerate}
        \item To eat braekfast is important. \\
              It is important to eat breakfast.
        \item I want to clean my room. 
        \item Cooking is fun. 
        \item The need a hous to live in. 
        \item I enjoy traveling around the world. 
        \item My dream is to be a soccer player. 
        \item I have many things to do. 
        \item Jake likes to play basketball.
    \end{enumerate}
    \section{문장 연결하기} % 접속사 
    \begin{enumerate}
        \item He is kind, smart, and honest
        \item Peter and I walked together. 
        \item My father smiled at me and asked.
        \item Which is your book, this or that? 
        \item Because it rained, we were not able to play basketball.
        \item I can finish my homework if he isn't here now.
        \item After Jack read the book, she gave it to her friend.
        \item I have to leave my house before 10 am. 
        \item I was watching TV when you called me. 
        \item He bought books but didn't read them. 
    \end{enumerate}
    \section{비교급}
    \begin{enumerate}
        \item I did better than yesterday.
        \item The food tasted nicer than what I made.
        \item He ran as fast as possible.
        \item Jack cannot get up as early as his friend. 
        \item I was as happay as yesterday. 
    \end{enumerate}
    \section{기타}

    \subsection{-s/-es 붙이기}
    \begin{enumerate}
        \item 대부분의 경우에 -s를 붙임 \\ girls, eggs, apples, friends, dogs, houses, ...
        \item 일반적으로 o, s, x, ch, sh로 끝나면 -es를 붙임 \\ boxes, tomatoes, glasses, dishes, buses, churches, ... \\
                주의) pianos, photos, videos, radios, audios
        \item y로 끝나면 y를 i로 바꾸고 -es를 붙임 \\ baby $\rightarrow$ babies, lady $\rightarrow$ ladies, party $\rightarrow$ parties, story $\rightarrow$ stories \\
                주의) key $\rightarrow$ keys, day $\rightarrow$ days, boy $\rightarrow$ boys, ...
        \item f나 fe로 끝나면 f/fe를 v로 바꾸고 -es를 붙임 \\ knife $\rightarrow$ knives, leaf $\rightarrow$ leaves, wife $\rightarrow$ wives, ... \\
                주의) roof $\rightarrow$ roofs, chief $\rightarrow$ chiefs, \dots
        \item 이외 불규칙형은 암기 
        
    \end{enumerate}
    \subsection{-d/-ed 붙이기}
    \begin{enumerate}
        \item 일반적으로 -ed를 붙임 \\ wanted, played, cleaned, helped, talked, failed, \dots
        \item e로 로 시작하는 동사에는 -d를 붙임 \\ liked, loved, lived, hated,\dots
        \item ``자음'' + y인 경우 y를 i로 고치고 -ed를 붙임 \\ study $\rightarrow$ studied, cry $\rightarrow$ cried, carry $\rightarrow$ carried, marry $\rightarrow$ married, \dots \\
                주의) play $\rightarrow$ played, obey $\rightarrow$ obeyed, enjoy $\rightarrow$ enjoyed, ... 
        \item 단모음 + 단자음 으로 끝나는 경우 자음을 하나 더 붙이고 -ed를 붙임 \\ stop $\rightarrow$ stopped, plan $\rightarrow$ planned, stop $\rightarrow$ stopped, drop $\rightarrow$ dropped, \dots
        \item 이외 불규칙형은 암기 
    \end{enumerate}

    \subsection{-ing 붙이기}
    \begin{enumerate}
        \item 대부분의 경우에는 -ing를 그냥 붙임. \\ going, doing, flying, pushing, talking \dots
        \item -e로 끝날 때에는 e를 없애고 -ing를 붙임. \\ have $\rightarrow$ having, make $\rightarrow$ making, love $\rightarrow$ loving, hate $\rightarrow$ hating, like $\rightarrow$ liking, drive $\rightarrow$ driving, \dots
        \item -ie로 끝날 때에는 ie를 y로 고치고 -ing를 붙임. \\ die $\rightarrow$ dying, lie $\rightarrow$ lying, \dots
        \item 단모음 + 단자음 ㅈ으로 끝날 때 자음 하나를 더 붙이고 ing를 붙임. \\ step $\rightarrow$ stepping, drop $\rightarrow$ dropping, run $\rightarrow$ running, get $\rightarrow$ getting, swim $\rightarrow$ swimming 
    \end{enumerate}
\end{document}