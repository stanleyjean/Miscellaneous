\documentclass[9pt, a4paper]{oblivoir}
\usepackage{fapapersize}
    \usefapapersize{210mm,297mm,20mm,20mm,25mm,25mm}
\usepackage{fancyhdr}
\usepackage{multicol}
\pagenumbering{gobble}
\fancypagestyle{firststyle}
{
    \pagestyle{fancy}
% \fancyhf{}
% \rhead{Overleaf}
\lhead{\textbf{\LARGE 교과서 본문}}
% \rfoot{Page \thepage}

}

\pagestyle{firststyle}

\usepackage[symbol]{footmisc}

\renewcommand{\thefootnote}{\alph{footnote}}

\newcommand{\symfootnote}[1]{%
\let\oldthefootnote=\thefootnote%
\stepcounter{mpfootnote}%
\addtocounter{footnote}{-1}%
\renewcommand{\thefootnote}{\fnsymbol{mpfootnote}}%
\footnote{#1}%
\let\thefootnote=\oldthefootnote%
}

% \setlength\parindent{0pt}
\usepackage{tabularray}

\begin{document}
    \section*{1과}
     
        \subsection*{Somin}
        \noindent
        Hello! I'm Somin. I'm 15 years old, and I live in Korea. Please tell me  about your favorite time of the day. You can show me some pictures.

        \subsection*{Diego}
        \noindent
        
        Hi, my name is Diego, and I live in Seville, Spain. My favorite time of the day is lunch time. My school usually ends around 2 p.m.,.
        On most days, my family gets together and has a big, long lunch. We usually have soup, vegetables, and meat. We also have a dessert like churros. After lunch, we usually take a siesta, a short\symfootnote{\textbf{siesta} 남부 유럽은 한낮에 일하기가 힘들 정도로 매우 더워서, 점심 식사를 한 뒤 2~3시간 정도 낮잠을 자는 풍습이 있다. 이것을 스페인에서는 `시에스타'라고 한다.} nap. 
        Both my father and I like to sleep under the tree in our garden. 

        \subsection*{Tabin}
        \noindent

        Hi! My name is Tabin, and I live near the Gobi Dessert in Mongolia. I'm happy when I ride my horse. Horses are important in our culture. Almost everyone can ride a horse in Mongolia. In fact, we say, ``We ride horses before we can walk.''

        I take good care of my horse. I often brush him and give him some carrots. I enjoy riding especially in the evening before the sunset. Then the sky is red, and everything is peaceful.

        \subsection*{Musa}
        \noindent
        
        Hi! I'm Musa, and I live in Nairobi, Kenya. My favorite time of the day is our running practice time. My friend, Tamu, and I are on the school's running team. I'm happiest when I run with Tamu. Our practice time isn't boring because we can see many animals. 

        Many runners from Kenya won races in the Olympics. I'm so proud of them Both Tamu and I want to be like them. \bigskip
        \vspace*{2cm}

        \subsection*{단어}

        % \begin{multicols}{3}
        %     xx years old: 나이가 xx살인

        %     favorite: 가장 좋아하는 
            
        %     show: 보여주다 
            
        %     usually: 대개, 일반적으로 

        %     around: 대략; 주위를 둘러서 

        %     together: 함께
            
        %     vegetable: 야채, 채소 

        %     nap: 낮잠 

        %     culture: 문화 
            
        %     ride: (자동차, 말 등을) 타다 

        %     peaceful: 평화로운 

        %     practice: 연습; 연습하다 

        %     be proud of: $~$를 자랑스러워 하다 


        % \end{multicols}

        

        \newpage

        \section*{2과}

        Hi, I'm Eddie parker. I live in Dallas, Texas. Now, my family and I are at the State Fair of Texas. The fair is over 130 years old, and it's the biggest fair in the USA. I'll show you around. Follow me!
    \\

        Look at the goats here! This is a goat show. My younger brother, Steve, entered it. The goats in the show don't have to be big, but they have to be healthy. They have to produce a lot of milk to win a prize. Steve took good care of his goat, Bonnie. Wow! Steve and Bonnie won a white ribbon, third price! I'm so proud of them! 
    \\

        Now, it's lunch time, so we're eating fair food. Mom and Dad are eating nachos and fajitas. They are Tex-Mex food. Tex-Mex food is a combination of food from Texas and Mexico. Dad's face is getting red because his fajita is too spicy. Steve and I are eating corn dogs. They taste great.
    \\

        Let's move on to the quilt contest Quilting has a long tradition. In the past, fabric was expensive. To make a blanket, people had to collect small peices of fabric and sew them together. 

        Grandma and her friends had a quilting bee\symfootnote{\textbf{bee} 사람들이 함께 모여서 공동의 일이나 놀이를 하는 모임을 bee라고 부른다.} every week. They had to work on the quilt for the contest for over six months. Look! It's very colorful and unique.
        \\


        The most exciting part of the day is riding the Texas Star. It's a tall Ferris wheel. Wow! Steve and I are now on the top. I'm scared, but the view is amazing! 

        I love living in Texas and going to the fair. Come to the fair some day! \bigskip

        \vspace*{6cm}

        \subsection*{단어}
        % \begin{multicols*}{3}
        %     fair: 박람회
            
        %     follow: 따라가다 
            
        %     goat: 염소
            
        %     enter: 들어가다, 입장하다
            
        %     produce: 생산하다
            
        %     prize: 상 
            
        %     combination: 조합
            
        %     spciy: 매운
            
        %     quilt: 퀼트
            
        %     tradition: 전통
            
        %     expensive: 비싼/cheap: 
            
        %     blanket: 담요, 이불 
            
        %     contest: 경연  
            
        %     colorful: (색깔 등이) 다채로운 
            
        %     unique: 독특한 
            
        %     exciting: 흥분되는, 흥미진진한, 신나는
            
        %     scared: 무서워하는, 겁 먹은
        
        %     view: 경치
        % \end{multicols*}

        \newpage


        \section*{3과}

        \textbf{Ms.Brown}: Hello, club members. As you know, this year's Environment Day is about upcycling.
        Before we talk about each group's event idea for that day, I want you to understand the meaning of ``upcycling.''
        Can anyone explain upcycling? \newline

        \noindent
        \textbf{Sumi}: Yes. The word ``upcycling'' is a combination of ``upgrade'' and ``recycling.''\newline

        \noindent
        \textbf{Eric}: Like recycling, upcycling is good for the environment. When you upcycle, you make new and better things from old things.\newline

        \noindent
        \textbf{Ms. Brown}: Good. Now, let's talk about each group's idea for the event. Let's start with Pei's group.\newline

        \noindent
        \textbf{Pei}: My group wants to hold a trashion show. ``Trashion'' is a combination of ``trash'' and ``fashion.'' 
        We'll use trash to make clothes. We want other stduents to become interested in upcycling through the show. \newline

        \noindent
        \textbf{Ms. Brown}: A trashion show sounds like fun! What abour your group, Eric?\newline

        \noindent
        \textbf{Eric}: My group is going to make musical instruments from old things. We'll make drums from old plastic buckets. 
        We'll also make a guitar from old boxes and rubber bands. We plan to play the instrumentsin a mini-concert. \newline

        \noindent
        \textbf{Ms. Brown}: Thank you, Eric. Now, let's hear from Sumi's group.\newline

        \noindent
        \textbf{Sumi}: My group wil make bags from old clothes. For example, we'll use blue jeans. Look at this bag.
        This was made by Hajun, one of our group members. Isn't it nice? 
        We'll make more bags and sell them on Environment Day. We're going to give all the money to a nursing home.\newline

        \noindent
        \textbf{Ms. Brown}: That's a great idea. Your ideas are all so creative. I want everyone to work hard for Environment Day. \newline

        \noindent
        \textbf{How to Make Blue Jeans Bag}

        You need: old blue jeans, sewing kit, scissors, pins and buttons 
        
        1. Cut off the legs of the blue jeans. 

        2. Sew the bottom together

        3. Make shoulder straps from one of the legs. 

        4. Sew the straps to the top of the jeans. 

        5. Decorate the big with pins and buttons.
    \subsection*{단어}
    % \begin{multicols*}{4}
    %     {\small%
    %     environment: 환경

    %     anyone: 아무나

    %     trash: 쓰레기 

    %     use: 사용하다 

    %     musical: 음악의; 뮤지컬 

    %     instrument: 악기; 기구

    %     bucket: 양동이, 버킷 

    %     sell: 팔다

    %     creative: 창의적인
        
    %     cut off: 끊다; 차단하다

    %     sew: 꿰매다 

    %     decorate: 장식하다
    %     }
    % \end{multicols*}
    \newpage

        \section*{4과}

        People once thought that only humans can use tools. Now, scientists are finding out that many animals can also use tools. \newline

        \textbf{Macaque Monkeys}

        If you go to a Buddhist temple in Lop Buri, Thailand, watch out for the Macaque monkeys\footnote{\textbf{Macaque Monkey}\\아시아와 북아프리카에 사는 원숭이의 한 종류이다.}. 
        They may come to you and pull out your hair. They use human hair to floss their teeth.
        If you are lucky, you may see female monkeys that are teaching flossing to their babies. 
        While the babies are watching, the female monkeys floss their teeth very slowly. This way, the baby monkeys learn to floss. \newline

        \textbf{Octopuses}

        People don't usually think that octopuses are smart. However, octopuses are very smart, and they can also use tools.
        They use coconut shells for protection. When they can't find a good hiding place, they hide under coconut shells. Some octopuses even store coconute shells for later use.
        They pile the coconut shells and carry them to use later. How smart! \newline

        \textbf{Crows}

        In Aesop's fable \emph{The Thirsty Crow}, a crow drops stones into a jar to raise the level of water. 
        You may think this is just a story, but it is not. Scientists who were studying crows did an experiment.
        They put a jar with water in front of a crow. A worm was floating on top of the water. However, the water level was low, so the crow could not eat the worm. The crow solved the problem just as in the fable. 
        It dropped stones into the jar. If you think this bird is special, you are wrong.
        Scientists did the same experiment with other crrows, and they all did the same, too. \bigskip 
        \vspace*{4cm}
        
        \subsection*{단어}

        % \begin{multicols*}{3}
        %     fable: 우화, 꾸며낸 이야기
            
        %     tool: 도구 

        %     scientist: 과학자

        %     Buddhist: 불교 

        %     temple: 사원 

        %     pull out: $~$을 떼어내다 

        %     floss: 치실; 치실질을 하다 

        %     octopus: 문어
            
        %     usually: 보통, 대개 

        %     protection: 보호 

        %     place: 장소 

        %     shell: 껍질 

        %     store: 저장하다; 상점 

        %     pile: 쌓다; 포개놓은 것 

        %     carry: 옮기다 

        %     experiment: 실험

        %     jar: 병 

        %     float: (물에) 뜨다

        %     solve: (문제 등을) 해결하다

        %     same: 같은
        % \end{multicols*}

    \newpage

    \section*{5과}

    Living without smartphones is difficult for many of us these days. However, unwise or
    too much use of smartphones can cause various problems. \newline
    
    \noindent
    \textbf{Are you a smombie?}

    All over the world, people are walking around like zombies. Their heads are down, and their
    eyes are on their smartphones. 
    We call such people smombies, smartphone zombies. 
    If you are a smombie, you can have various safety problems. 
    You may not see a hole in the street, so you may fall and get hurt. 
    You may get into a car accdient, too. So what can you do to prevent these problems? 
    It's simple. Do not look at your smartphone while you are walking! \newline

    \noindent
    \textbf{Do you have dry eyes or text neck?}

    Smartphones can cause various health problmes. One example is dry eyes. 
    When you look at your smartphones, you do not blink often. Then your eyes will feel dry. 
    
    Another problem you can have is neck pain. When you look down at your smartphone, the stress on your neck increases.
    Too much use of your smartphone, for example, too much texting, can cause neck pain.
    We call this text neck.

    Here are some tips for these problmes. For dry eyes, try to blink often. For text neck, move your smartphone up to your eye level. You can also do some neck stretching exercises.\newline

    \noindent
    \textbf{How do you feel when you don't have your smartphone with you?}

    Do you feel nervous when your smartphone is not around? Do you feel sad when you check your smartphone and there is no text message? 
    If your answers are ``yes,'' you may have smartphone addtiction. There are various things you can do to prevent this. 
    For example, turn off your smartphone during meals or meetings. You coan talk to people instead of texting them. 
    
    \vspace*{9cm}
    \subsubsection*{단어}
    % \begin{multicols*}{3}
    %     jfdkaljfkls
    % \end{multicols*}

    \newpage

    \section*{6과}

    We often find different painting with the same subject. An different paintings with the same subject.
    AN example is \emph{The Flight of Icarus} by Henri Matisse and \emph{The Fall of Icarus} by Marc Chagall. They are both about the same Greek myth.

    \subsection*{The Greek Myth of Icarus}

    Daedalus was a great inventor. King Minos liked Daedalus' work so much that he wanted to keep Daedalus with him forever. 
    Daedalus, however, tried to leave, so the King kept him and his son, Icarus, in a tall tower.
    Daedalus wanted to escape. 

    One day, Daedalus saw birds flying. ``Wings! I need wings!'' he shouted. Daedalus then gathered bird feathers and glued them together with wax.
    When the wings were ready, he warned his son, ``Don't fly too cloes to the sun. The was will melt.''

    Daedalus and Icarus began to fly. Icarus was so excited that he forgot his father's warning. He flew higher and highger, and the wax began to melt. 
    ``Oh, no! I'm falling,'' Icarus cried out. Icarus fell into the sea and died.

    \subsection*{Two Different Paintings}

    Matisse and Chagall both deal with the same subject in their paintings, but they are different. 

    % picture 
    FIrst, in Matisse's painting, you can see Icarus flying, but in Chagall's painting, the boy is falling. This difference comes from the different ideas that the two painters had.
    Matisse thought that Icarus was brave and adventurous. In contrast, Chagall thought that Icarus was foolish.

    Second, Matisse's painting is very simple, but Chagall's painting has many details. In Matisse's painting, there are only Icarus and some stars. Furthermore, Icarus' body has just a simple outline. In contrast, Chagall painted many people and houses in addition to Icarus. This difference comes from the different painting styles of the two painters. \newline

    Whose painting do you like more? People will have different answers because they omay see the same thing in different ways. 
    \vspace*{9cm}
    \subsubsection*{단어}
    % \begin{multicols*}{3}
    %     fjdiowarjw
    % \end{multicols*}

    \newpage

    \section*{7과}

    Rada lived on a little world, far out in space. She lived there with her father, mother, and brother Jonny. 
    Rada's father and other people worked on spaceships. Only Rada and Jonny were children, and they were born in space. \newline

    One day, Dad told Rada and Jonny, ``We're going back to Earth tomorrow.''

    Rada and Jonny looked at Dad in surprise and floated towards him. Rada asked Dad, ``What's like on Earth?''

    ``Everything is different there. For example, the sky is blue,'' answered Dad.

    ``I've never seen a blue sky,'' said Jonny.

    ``The sky is always black here,'' said Rada.

    ``You don't have to wear your big heavy space suits because there is air everywhere.
    It's also hard to jump there because Earth pulls you down,' said Dad.

    ``What else?'' asked Rada.
    
    ``There are hills, and they are covered with soft green grass.
    You can roll down the hills,'' answered Mom.

    ``Dad, have you ever rolled down a hill?'' asked Rada.
    
    ``Yes, it's really amazing!'' answered Dad.

    Jonny was thirsty, so he opened a milk container and shook it. The milk floated in the air and formed balls . Jonny swallowed the balls. 
    
    ``Jonny, if you drink milk that way on Earth, you'll get wet,'' said Mom. \newline

    Later that night, Rada and Joinny talked a long time about Earth. IT was exciting to think about all the new things they were going to see and do.
    There was one new thing Rada and Jonny really wanted to do.
    They thought about it all night and didn't tell Mom and Dad about it. It was their secret.

    The next day, Rada's family got on a spaceship. 

    ``It's going to be a long trip,'' said Mom.
    
    ``That's alright. I'm so excited!'' said Rada. \newline

    The spaceship finally landed. 

    ``Dad, it's difficult to walk on Earth,'' said Rada.

    ``I know. Earth is pulling you down,'' said Dad.

    Rada and Jonny couldn't float anymore. That was the first new thing. 

    ``What's that sounds?'' asked Rada.

    ``A bird is singing,'' said Mom.

    ``I've never heard a bird sing,'' said Rada.

    ``And I've never felt the wind,'' said Jonny. 

    These were all new things. 

    Rada and Jonny ran up the nearest hill. At the top, they looked at each other and laughed.
    Then they lay down on the soft green grass and rolled down the hill. That was their secret!.

    ``This is the best new thing of all!'' shouted Rada and Jonny. And they ran up to the top of the hill again.

    \vspace*{4cm}
    \subsubsection*{단어}
    \newpage
    % \begin{multicols*}{3}
    %     fdaf
    % \end{multicols*}

    \section*{8과}

    On May 27, 2011, 297 boooks of \emph{Uigwe}\footnote{\textbf{의궤}: 조선 왕실의 중요한 행사와 건축 등을 글과 그림으로 상세하게 정리한 기록}, a collection of royal books the French army took in 1866\footnote{\textbf{병인양요}: 1866년 프랑스 함대가 강화도에 침범한 사건},  came back to Korea. 
    The person behind this return is Dr. Park Byeong-seon, a historian who spent her whole life searing for Korean national treasures abroad. 

    \subsection*{Can you tell me how you became interested in \emph{Uigwe}?}

    \begin{itemize}
        \item[Dr. Park:] I studied history in college. I went to France to continue my studied in 1955. As you know, the French army took many of our national treasures in 1866. I wanted to find them while I was studying there. \emph{Uigwe} was one of them.
    \end{itemize}

    \subsection*{You found 297 books \emph{Uigwe} in the National Library of France, in Paris. Please tell me how you found them..}

    \begin{itemize}
        \item[Dr. Park:] As soon as I became a researcher at the National Library in 1967, I began to look for \emph{Uigwe}. 
        After 10 years,in 1977, I finally found the books. I think I looked at more than 30 million books.
    \end{itemize}

    \subsection*{I'm sure you were very excited when you found the books.}
    \begin{itemize}
        \item[Dr. Park:] Yes, I wa,s but more difficulties were waiting for me. I thought that the books should be returned to Korea, but my bosses at the library didn't like that idea.
        They even thought that I was a Korean spy and fired me. After that, I had to go to the library as a visitor, so it was not easy to do research on \emph{Uigwe}. 
        However, I didn't give up. For more than ten years, I went to the library every day to finish my reseawrch. 
        I wanted to show people the value of \emph{Uigwe}.
    \end{itemize}

    \subsection*{The reulsts of your research were published as a book in Korea in 1990. Many Koreans became interested in \emph{Uigwe} because of your book.}

    \begin{itemize}
        \item[Dr. Park:] Yes. In 1992, the Korean government asked the French government for its returen and, finally, the 297 books are here now.
        

    \end{itemize}
    
    \subsection*{Before I finish this interview, I'd like to ask you about \emph{Jikji}, a book that changed the history of painting.}

    \begin{itemize}
        \item[Dr. Park:] I found it in my first year at the library. I knew right away that it was very special. 
        I worked hard to prove its value and finally succeeded. At a book exhibition in Paris in 1972, \emph{Jikji}\footnote{1972년 ``세계 도서의 해 기념 도서 전시회''에서 ``직지''는 금속 활자 (movable metal type)로 인쇄된 가장 오래된 책으로 알려져 있던 독일의 구텐베르크 성경보다 78년 앞서 인쇄되었음을 인정받았다.} was displyed as the oldest book in the world that was printed with movable metal type.
    \end{itemize}

    \subsection*{Dr. Park, thanks to your hard work, \emph{Jikji} and \emph{Uigwe} were found, and all Koreas thank you for that.}

    \begin{itemize}
        \item[Dr. Park:] I hope people will become more interested in our naitonal treasures abroad and work for their return.
    \end{itemize}
    \vspace*{0.5cm}
    \subsubsection*{단어}
\end{document}