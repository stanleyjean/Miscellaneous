\documentclass[10pt, a4paper]{oblivoir}
% \usepackage{apacite}
\usepackage[style = apa, backend = biber]{biblatex}
\usepackage{fapapersize}
    \usefapapersize{210mm,297mm,20mm,20mm,25mm,25mm}
\pagenumbering{gobble}

\setcounter{tocdepth}{5}
\setcounter{secnumdepth}{5}

\date{\large 2022~-~2학기}
\title{\Huge 일반사회논리및논술 실행연구 계획서\\
    \LARGE 수업을 통한 학생의 학습 효용감 제고 방안 탐구 및 관찰 방법 연구}
\author{\Large 2020127018 사회학과 손민우}

\addbibresource{ilsanon_paper_citation.bib}  

\begin{document}
    \maketitle
    \tableofcontents*
    \pagebreak

    \section{선행 연구 분석}
    \cite{정민수2018교육과정}는 실행연구 방법론을 바탕으로 `교육과정-수업-평가-기록'의 일체화 수업 연구 과정을 구성하였다. 
    또한 동일한 글에서 `교실 수업에서의 일체화를 위해 교사는 수업을 계획하고 실행하는 과정에서 \textbf{목표 지향적}이어야 하며, 
    분명한 학습 목표를 염두에 두고 계획된 학습목표와 부합된 전략과 활동을 사용해야 한다"고 이야기 한다.\footnote{원문에서는 굵은 글씨 처리가 없었고 본 계획서에서는 의미를 강조하고자 추가함} 동일한 글에서 평가를 위한 준거 기준으로써 
    ``구조화"와 ``분류화"를 제시하는데, 구조화는 ``비대칭적인 권력관계" 속에서 상징화된 정체성의 틀에 맞추어 의사소통이 진행되는 점을 
    이야기하고 분류화는 ``형성된 권력 구조"를 바탕으로 객체를 나누는 것을 의미한다. 이를 바탕으로, 수업 내부에서 교사와 학생 간의 
    의사소통이 진행되는 과정 속에서 구조화와 분류화의 정도를 제시함으로써 수업의 진행 방향을 이야기해볼 수 있을 것이며, 결정된 진행 방향은
    앞서 말한 학습 목표와 연관된 내용을 잘 조직화 할 수 있도록 구성하기 위해 어떠한 점이 요구되는지 검증할 필요가 있다고 느꼈다. 

    \cite{강대현2013사회과}은 교사의 전문성 발달에 관해 판단할 수 있는 통합적인 틀을 제시한다. 골자는 교사의 수업 내용 전달력, 교과의 전문지식 보유 여부,
    수업에 관한 성찰 정도 등이 종합적으로 교사가 가지는 역량에 영향을 미친다는 것이다. 중등교육에서 교사의 역햘은 단순히 교과내용 전달에 그치는 것이
    아니라 학생 생활 전반과 진로 진학 지도까지 포함한다는 점을 고려해보았을 때, 교사는 교과 전문성에 더해 학생의 일상적이거나 특수한 갈등, 고민 상황에 대해서 
    올바르게 조언할 수 있고 `성찰하기 위한 틀'을 제시할 수 있어야 한다. 따라서 학생들이 수업 과정에서 가질 수 있는 수업과 직/간접적으로 관련되는 
    여러 고민들을 교사가 이해할 수 있는 방법은 없을지에 대해 생각해 볼 필요가 있을 것이다. 비슷한 맥락 속에서,
    \cite{한광웅2010사회과}은 ``예비교사의 수업전문성은 수업의 의미와 교사의 정체성 및 역량에 대해 근본적인 수준에서 반성하고,
    이를 수업실천의 맥락에서 실행할 수 있는 능력과 개방된 마음가짐을 가리킨다''고 하였다. \cite{leeEnglish}은 ``\dots 실행연구 주체로서 교사는 
    현장에서 학생들이 느끼는 문제점을 스스로 파악하고 주도적으로 해결방안을 탐색하여야 한다'', ``학생들이 진정으로 필요로 하는 해결책을 제시하기 위해서는 
    학생들이 당면하는 문제를 규명하는 행위가 실행연구 과정에서 시작점이 되어야 한다.''고 언급한다. 마찬가지로, 수업관련 정체성은 교사로서 아이들에게 무엇을 가르치고자 하는지, 수업의 목적을 어디
    에 두며 그것이 교사 자신과 어떻게 관련되는지를 탐색하는 것을 의미한다.(\cite{강지영2011국내})\\
    정리하자면 교사는 능동적으로 학생이 가지고 있는 문제를 파악하려 노력하는 동시에, 제 3자의 입장으로서가 아니라 학생 본연의 입장에서 
    필요로 하는 해결책을 고민해볼 필요가 있다. 

    사회과 연구는 연구 주제의 측면에서 `수업모형 및 교수·학습방법’ 토픽과 `교과서 및 교육과정’ 토픽에
     집중된 경향이 나타났다.(\cite{김재우2019텍스트}) 실행연구의 목적이 전문성 발달뿐만 아니라 연구자 개인의 
    내면적 성장과 사회구조 개선을 위한 것으로 다양화될 필요가 있다.(\cite{강지영2011국내}) \newline

    앞선 논의를 종합하면 아래와 같다. 
    \begin{itemize}[$\ast$]
        \item 교사 스스로 학생들의 반응과 행위를 확인하면서 교육 과정에 알맞게 학습 목표를 구성 및 조직하는 방법에 관해서 연구가 필요하다.
        \item 교사들이 학생들이 진정으로 필요한 요소를 어떻게 하면 찾아낼 수 있을 것인지 파악해야한다. 
        \item 단순히 교육과정을 잘 따라서 충실하게 ``학습''하도록 돕는 것에서 나아가 개별적인 학생의 학습 ``과정''에 대한 탐구가 요구된다. 
        \item 학생들과 관련된 교사의 ``탐구''에 있어서 교사가 가져야할 태도나 관찰 방법은 어떤 것이 적절할지 생각해보는 것이 바람직하다. 
    \end{itemize}

    이를 바탕으로 연구 질문을 구성하였다. 
    

    \section{연구 질문}
    \begin{enumerate}
        \item 어떻게 수업을 통해 학생들이 각자 학습 최적 경로를 찾을 수 있도록 도울 수 있을까?  \\ 읽기 자료를 통한 학습, 발표나 토의를 이용한 학습을 통해 적절한 학습 방법과 자료를 파악한다. 
        \item  이를 바탕으로 수업에서 나타나는 학생들의 상호작용이 실질적인 내용 학습에 어떠한 관계를 가질까? \\  수업 중에 제시되는 과제들을 통해 학생들이 갖고 있는 수업의 이해도와 사고의 깊이를 확인한다. 
        \item 다양한 관찰 방법을 통해 수업 중의 학생과 교사, 학생과 학생의 상호작용을 파악하면서 가장 유의미한 결과를 얻을 수 있는 관찰 방법은 어떤 것이 있을지 파악한다. 
    \end{enumerate}

    \section{연구 계획}

        \subsection{수업 계획}
        \textbf{학습목표}: 사회적 소수자에 대한 개념과 특성을 이해하고, 관련된 사회적 이슈에 대해 논의하며 학생 스스로 
        사회적 소수자와 관련된 문제에 대해 자신의 의견을 종합하여 개진할 수 있다. 
            \subsubsection{도입}
            \begin{itemize}[-]
                \item 출석을 확인한다.
                \item 사회적 소수자 단원이 갖는 학습의 의의를 학생들에게 제시한다.
            \end{itemize}
            \subsubsection{전개}
            \begin{enumerate}
                \item 교과서 내용 학습
                \begin{itemize}[-]
                    \item 사회적 소수자의 개념과 예시를 흥미를 유발할 수 있는 영상자료와 함께 이해할 수 있도록 한다. 
                    \item 교과서의 학습활동을 통해 사회적 소수자에 대한 기본적인 개념을 학습할 수 있도록 하고, 이해 여부에 관해 발표를 유도한다.
                \end{itemize}
                \item 토론 학습
                \begin{itemize}[-]
                    \item 사회적 소수자의 상황에 대한 기사 자료를 읽기 자료로써 제시한다. \\ 논란이 있을 수 있으며 찬반의 여지가 나뉘는 내용을 제시하여 생각을 열 수 있도록 한다. 
                    \item \textbf{읽기 자료와 발표 및 토의를 함께 활용하는 동시에, 이를 각각 제시하여 효과를 확인 해보고자 한다.}
                \end{itemize}
                \item 심화 및 확장 학습
                \begin{itemize}[-]
                    \item 교과서에서 제시되어 논의의 바탕으로 이용한 사회적 소수자의 개념에 대해 재고해볼 수 있는 기회를 가질 수 있도록 한다. 
                    \item 사회 정체성이 다원화 되고 있다는 점을 고려하여, 학생의 단편적인 사회적 소수자 개념의 확장 가능성에 대해 중점적으로 논의한다. 
                \end{itemize}
            \end{enumerate}
        \subsubsection{평가 및 정리}
        \begin{itemize}[-]
            \item 학습지를 통해 본 수업의 내용을 정리하고 학생 스스로 생각과 기본 개념을 정리할 수 있도록 한다. 
            \item 찬반의 격화된 논의가 필요한 주제가 아니라 ``문제''를 해결하는 방안을 고민해봄으로써, 수업시간에 배운 내용을 실제 현실의 내용과 연결지어 생각할 수 있도록 한다. 
            \item 학생이 수업 내용을 받아들이고 공감하여, 최종적으로 수업의 가치를 스스로 파악하여 효용감을 가질 수 있는지 확인해보는 것이 필요하다. 
        \end{itemize}

        \subsection{자료 수집 계획}
        가능하다면 양적, 질적 자료 모두를 이용해서 결과를 도출하는 것이 조금이라도 더 유의미한 결과를 낼 수 있을 것이다. 추가적으로 자료 수집에서
        중점적으로 고려해야할 점은 학생이 수업 과정 중이나 수업 후에, 학생들이 ``원하는 바''를 이해하고 짚어낼 수 있도록 돕는 요소를 파악해야한다는 것이다.
        \subsubsection*{양적 자료}
        수업 대상의 학생 수가 제한적이므로, 양적 자료를 이용해서 많은 결과값을 얻을 수 없다는 점을 고려하면, 학생들이 
        수업을 겪으며 느끼는 감상을 간단히 수치화하여 수집하는 것이 유의미하다고 판단했다. 따라서 양적자료는 수업 직후에 
        학생이 수업의 각 과정에서 느꼈던 ``향상감''이나 ``어려움'' 등을 측정할 수 있는 점수를 수집하고자 한다. 
        \subsubsection*{질적 자료}
        수업 전, 중, 후에 관찰을 통해서 학생들이 수업에 집중하는 정도나 수업에 열의 등을 확인하고자 한다. 수업 중에 
        제시되는 학습지에 대한 충실도 여부와 앞서 이야기된 양적 자료를 간단히 비교해보고자 한다. 수업을 전개해나가는 과정 속에서 
        학생의 사고가 심화되는지, 혹은 학생이 수업의 내용을 받아들이지 못하고 사고 자체도 정체가 되는지 등 다층적인 면에서 학생이 
        이용한 학습지를 분석하고자 한다. 

\pagebreak



% 개별 인용 \cite{bibi_name}
 %   \bibliographystyle{apacite}
 %   \bibliography{ilsanon_paper_citation.bib}
\printbibliography

\end{document}